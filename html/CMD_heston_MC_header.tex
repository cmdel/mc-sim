
    
\section*{Monte Carlo Simulation of the Heston Model}

\begin{par}
Calculate the value of a European call vanilla option using the Heston model for stochastic volatility
\end{par} \vspace{1em}

\subsection*{Contents}

\begin{itemize}
\setlength{\itemsep}{-1ex}
   \item USAGE
   \item INPUTS
   \item OUPUTS
   \item REFERENCES
   \item ATTRIBUTION
\end{itemize}


\subsection*{USAGE}

\begin{par}
[call,std\_err]=MCheston(So, r, V0, xi, theta, kappa, K, Maturity,                 NoSteps, NoPaths)
\end{par} \vspace{1em}


\subsection*{INPUTS}

\begin{verbatim}S0     - Current price of the underlying asset.
r      - Annualized continuously compounded risk-free rate of return
         over the life of the option, expressed as a positive decimal
         number.
q      - Annualized continuously compounded yield rate\end{verbatim}
\begin{verbatim}V0     - Current variance of the underlying asset
xi     - volatility of volatility
theta  - long-term mean
kappa  - rate of mean-reversion\end{verbatim}
\begin{verbatim}strike      - Vector of strike prices of the option
Maturity    - Time to expiration of the option, expressed in years.
NoSteps     - Number of time steps per path
NoPaths     - Number of paths (Monte-Carlo simulations)\end{verbatim}


\subsection*{OUPUTS}

\begin{verbatim}call\_prices     - Prices (i.e., value) of a vector of European call options.
std\_err         - Standard deviation of the error due to the Monte-Carlo
                  simulation:
                  (std\_err = std(sample)/sqrt(length(sample)))\end{verbatim}


\subsection*{REFERENCES}

\begin{par}
[AN06] Andersen, Leif. 2006. “Efficient Simulation of the Heston Stochastic Volatility Model 1 Introduction.”
\end{par} \vspace{1em}


\subsection*{ATTRIBUTION}

\begin{par}
Christos  Delivorias The University of Edinburgh August 2012
\end{par} \vspace{1em}


